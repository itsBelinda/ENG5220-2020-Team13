This page highlights the means by which the device accounts are managed i.\+e. the device/\+Account... files.

\subsection*{\hyperlink{class_account}{Account}}

In order for the device to be able to determine whether or not its within a fence, local representations of the defined virtual fences (alongside their attributes) must be present. Moreover, in order for a parent / guardian to be notified in the event the aforementioned fences are breached, contact details too must be stored locally.

The \hyperlink{_account_8h}{Account.\+h} / .cpp files, consequently, encapsulates the said copies of fences and contact details, permitting the \hyperlink{_monitor_8h}{Monitor.\+h} / .cpp class (thread) to invoke / obtain required information, namely\+:


\begin{DoxyItemize}
\item Obtaining the list (vector) of fences in order to invoke the appropriate functions to determine whether or not the latitude and longitude is within that fence.
\item Obtaining the contact details for a parent / guardian in the event a fence is breached.
\end{DoxyItemize}

\subsubsection*{Instantiation}

{\itshape Due to recent global events, code pertaining to the definition of the fences / accounts via the web A\+PI was not achieved. Thus, a status account definition is provided -\/ should be redefined / removed from code. However, ideally, the user would utilise \href{https://oauth.net/2/device-flow/}{\tt O\+Auth 2.\+0 device flow} in order to grant the Bee\+Safe device an A\+PI token to access J\+S\+ON web A\+PI and obtain the account definitions in a secure manner; the u-\/\+Blox device is capable of supporting secure connections via http(s) using S\+SL. This would be performed during the initial setup phase, analogous to activating a smart tv or any other input-\/limited device.}


\begin{DoxyCode}
\hyperlink{class_account_a14f13a7a2e1d69d4b78c804603d69e1b}{Account::Account}(std::string &name, std::vector<Contact*>& contacts,
                 std::vector<Fence*>& fences)
\{
    this->name = \hyperlink{class_account_a586e2c3461c5231eacf7c96851024a75}{name};
    this->contacts = \hyperlink{class_account_aa4f77abd7c44f2a70b0cff8088e3491f}{contacts};
    this->fences = \hyperlink{class_account_ad92a9e8008371f34da06cd416a716fa1}{fences};
\}
\end{DoxyCode}


Upon start-\/up, the device instantiates the required components (comms/ and monitor/), however, the \hyperlink{class_account}{Account} need not be present for the device to successfully initialise. Instead, the device does not start the monitor thread, but attempts to download the required J\+S\+ON account definition ({\itshape to be implemented}). Alternatively, if a local copy of the account is present -\/ presuming all prior components initialise successfully -\/ the device can start the monitor thread.

\subsubsection*{Builder}

Due to various uncertainties surrounding the environment in which the code operates, namely the state of the serialised Account.\+json file, the \hyperlink{_account_builder_8h}{Account\+Builder.\+h} / .cpp utilises a \href{https://sourcemaking.com/design_patterns/builder}{\tt builder design pattern} -\/ an all or nothing approach. If, for whatever reason, the file is invalid / corrupt, it will not be serialised into an \hyperlink{class_account}{Account} instance. In which case, the device {\itshape would} follow the most appropriate procedure in attempting to re-\/download the account definition from the web A\+PI.

The \hyperlink{class_account}{Account}, \hyperlink{class_contact}{Contact}, \hyperlink{class_fence}{Fence}, \hyperlink{class_poly_fence}{Poly\+Fence} and \hyperlink{class_round_fence}{Round\+Fence} instances are all serialised according to a hierarchical structure within the Account.\+json file; utilised due to it providing a minimalistic file size while whilst being supported by the H\+T\+T\+P(s) protocols. However, each instance is responsible for serialising a J\+S\+ON element, which in turn is appended to an object / array element, in turn composing the Account.\+json file. Serialisation is achieved by employing \href{https://github.com/microsoft/cpprestsdk}{\tt Microsoft\textquotesingle{}s open-\/source cpprestsdk library.} due to providing capabilities for asynchronous invocation for web A\+PI\textquotesingle{}s, J\+S\+ON serialisation, task management, and more. 