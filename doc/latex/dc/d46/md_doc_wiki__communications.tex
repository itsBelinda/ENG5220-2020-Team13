The following page highlights various properties surrounding the comms/ package i.\+e. the communications stack implemented which includes \hyperlink{_comms_8h}{Comms.\+h} / .cpp, \hyperlink{_u_blox_8h}{U\+Blox.\+h} / .cpp, and \hyperlink{_u_art_8h}{U\+Art.\+h} / .cpp.

\section*{The Communications Interface}

The communications (or comms) interface is implemented analogous to the I\+SO networking stack, in which levels are abstracted from one another; lower levels implement low-\/level interface code, whilst higher levels implement an A\+P\+I-\/like interface. The following image depicts the communications stack\+:



As stated, each level is responsible for a separate task, namely\+:
\begin{DoxyItemize}
\item The {\bfseries \hyperlink{_comms_8h}{Comms.\+h} / .cpp} provides a synchronised A\+PI interface (via mutex) to the low-\/level communications code; multiple threads may utilise the communications package simultaneously e.\+g. \hyperlink{_monitor_8h}{Monitor.\+h} / .cpp for monitoring and \hyperlink{_bee_safe_8h}{Bee\+Safe.\+h} / .cpp for management (downloading updates, accounts, etc.). Moreover, this prevents the need to implement / manage locking of the relatively complex interoperability of the underlying levels, thereby yielding a considerably more coherent implementation / structure. Additionally, locking mechanism aids and protects the underlying stacks finite-\/state-\/based implementation (internally, the underlying stack is responsible for entering and leaving a set of read / write states in order to process responses from a command).
\item {\bfseries \hyperlink{_u_blox_8h}{U\+Blox.\+h} / .cpp} specialises in configuring the underlying device via \href{https://github.com/itsBelinda/ENG5220-2020-Team13/wiki/uBlox-and-CellLocate}{\tt AT commands} in addition reading and writing the said commands via the U\+A\+RT (Universal Asynchronous Receiver/\+Transmitter) interface.
\item Finally, {\bfseries \hyperlink{_u_art_8h}{U\+Art.\+h} / .cpp} is responsible reading and writing from and to the U\+A\+RT interface, respectively, on a byte level in addition to configuring the serial interface via \href{http://man7.org/linux/man-pages/man3/termios.3.html}{\tt termios}.
\end{DoxyItemize}

\subsection*{Implementation}

The following section addresses implementations with respect to the aforementioned communications stack. \subsubsection*{\hyperlink{_comms_8h}{Comms.\+h} / .cpp}

As stated, the \hyperlink{_comms_8h}{Comms.\+h} / .cpp files are responsible for providing concurrent synchronised access to the underlying interface. Given that only one device exists, attempting to invoke commands concurrently {\itshape may} result in corrupting the \hyperlink{_u_blox_8h}{U\+Blox.\+h} / .cpp result buffer (by clearing it or mixing \href{https://github.com/itsBelinda/ENG5220-2020-Team13/wiki/uBlox-and-CellLocate}{\tt AT command response formats}) or aborting pending commands.


\begin{DoxyCode}
\textcolor{keywordtype}{bool} \hyperlink{class_comms_aa0519d3ed2d5bd6aad60101080ac2de7}{Comms::init}()
\{
    \textcolor{comment}{// Lock and initialise the comms interface.}
    \hyperlink{class_comms_a21df861b1202573e4cd0cb5666d638fe}{mtx}.lock();
    \textcolor{keywordtype}{bool} initialised = \hyperlink{class_comms_ac64dea134b116147e5441172346dbd6c}{uBlox}.\hyperlink{class_u_blox_a34c2f507ff3bbd21b9aea788a015527a}{init}();
    \hyperlink{class_comms_a21df861b1202573e4cd0cb5666d638fe}{mtx}.unlock();

    \textcolor{comment}{// Whether or not the comms was successfully initialised.}
    \textcolor{keywordflow}{return} initialised;
\}
\end{DoxyCode}
 Due to functional dependencies associated with the communications interface, the \hyperlink{_bee_safe_8h}{Bee\+Safe.\+h} / .cpp manager attempts to initialise the \hyperlink{_comms_8h}{Comms.\+h} / .cpp interface -\/ and in turn the entire stack -\/ three times; the prior code-\/snippet posits the \hyperlink{_comms_8h}{Comms.\+h} / .cpp initialisation function. In doing so, the entire stack is re-\/initialised, once again, adopting an all or nothing approach. If, however, the interface fails to initialise, the program is terminated as we are unable to do {\itshape anything}.


\begin{DoxyCode}
\hyperlink{class_comms}{Comms}* \hyperlink{class_bee_safe_manager_a28306d7ccf7136a6086d666f4ebb6566}{BeeSafeManager::initComms}()
\{
    \textcolor{keyword}{auto} \hyperlink{class_bee_safe_manager_a80b19afbb679d08be14d67a45447f9e1}{comms} = \textcolor{keyword}{new} \hyperlink{class_comms}{Comms}();

    \textcolor{comment}{// Try numerous times to initialise the comms interface.}
    \textcolor{keywordtype}{bool} init = \textcolor{keyword}{false};
    \textcolor{keywordtype}{short} tries = 0;
    \textcolor{keywordflow}{do} \{

        \textcolor{comment}{// Attempt to initialise the interface.}
        tries++;
        std::cout << \textcolor{stringliteral}{"Comms initialisation attempt "} << tries << \textcolor{stringliteral}{" / "} << 
      \hyperlink{_bee_safe_8cpp_aa5860d80bbb4527d5d2275aacfce65f7}{INIT\_COMMS\_TRIES} << \textcolor{stringliteral}{"..."} <<  std::endl;
        init = \hyperlink{class_bee_safe_manager_a80b19afbb679d08be14d67a45447f9e1}{comms}->\hyperlink{class_comms_aa0519d3ed2d5bd6aad60101080ac2de7}{init}();
        \textcolor{keywordflow}{if} (init) \{
            \textcolor{keywordflow}{break};
        \} \textcolor{keywordflow}{else} \{
            std::cerr << \textcolor{stringliteral}{"... comms initialisation attempt "} << tries << \textcolor{stringliteral}{" / "} << 
      \hyperlink{_bee_safe_8cpp_aa5860d80bbb4527d5d2275aacfce65f7}{INIT\_COMMS\_TRIES} << \textcolor{stringliteral}{" failed."} << std::endl;
        \}

    \} \textcolor{keywordflow}{while} (tries < \hyperlink{_bee_safe_8cpp_aa5860d80bbb4527d5d2275aacfce65f7}{INIT\_COMMS\_TRIES});

    \textcolor{comment}{// We failed to initialise the interface.}
    \textcolor{keywordflow}{if} (!init) \{
        \textcolor{keyword}{delete} \hyperlink{class_bee_safe_manager_a80b19afbb679d08be14d67a45447f9e1}{comms};
        \textcolor{keywordflow}{return} \textcolor{keyword}{nullptr};
    \}

    \textcolor{comment}{// Successfully return the comms instance.}
    \textcolor{keywordflow}{return} \hyperlink{class_bee_safe_manager_a80b19afbb679d08be14d67a45447f9e1}{comms};
\}
\end{DoxyCode}


\subsubsection*{\hyperlink{_u_blox_8h}{U\+Blox.\+h} / .cpp}

The \hyperlink{_u_blox_8h}{U\+Blox.\+h} / .cpp files are arguable the most important files within the communication stack, defining both the commands and the means by which they are interpreted (due to differing formats). First and foremost, however, the \hyperlink{class_u_blox}{U\+Blox} layer is responsible for \href{https://github.com/itsBelinda/ENG5220-2020-Team13/wiki/uBlox-and-CellLocate}{\tt configuring the underlying u-\/\+Blox device} during the initialisation phase. During the said phase, the layer performs the following operations, terminating (returning) in the event of a failure at any stage\+:


\begin{DoxyItemize}
\item First, invoking the \hyperlink{_u_art_8h}{U\+Art.\+h} / .cpp initialisation functions to establish a serial connection with the device. Failing to do so will result in a premature termination (return) of the \hyperlink{class_u_blox}{U\+Blox} initialisation function indicating that the device could not be initialised.
\item Checking whether the S\+IM card has been registered. If not, automatic S\+IM card registration is started.
\item By definition, successful registration of the S\+IM card should yield a state in which G\+P\+RS is considered to be \textquotesingle{}attached\textquotesingle{}. If, however, the command could not be performed (maybe the device is not present?), the function returns i.\+e. \hyperlink{_u_blox_8h}{U\+Blox.\+h} / .cpp failed to communicate with the u-\/\+Blox device.
\item \href{https://github.com/itsBelinda/ENG5220-2020-Team13/wiki/uBlox-and-CellLocate}{\tt Establishing an internet connection (P\+SD)}.
\item Setting the message mode to T\+E\+XT so that text messages could be sent to any of the defined contacts.
\item Setting the location scan mode -\/ used for obtaining the location (latitude and longitude) of the device.
\end{DoxyItemize}

The following code snippet posits the initialisation function (without the messages printed to the terminal)\+: 
\begin{DoxyCode}
\textcolor{keywordtype}{bool} \hyperlink{class_u_blox_a34c2f507ff3bbd21b9aea788a015527a}{UBlox::init}()
\{

    \textcolor{comment}{// Initialise the UART device and interface.}
    \textcolor{keywordflow}{if} (!\hyperlink{class_u_blox_a034c0463d1c199d094d657c8ebb151e8}{uArt}.\hyperlink{class_u_art_a51adaa81c08d92599768c0303e5abc94}{init}()) \{
        \textcolor{keywordflow}{return} \textcolor{keyword}{false};
    \}

    \textcolor{comment}{// Check if the SIM has been registered.}
    \textcolor{keywordtype}{bool} simRegistered = \textcolor{keyword}{false};
    \textcolor{keywordflow}{if} (!\hyperlink{class_u_blox_a1889c2b9bb6087bc939bd2a27b68623b}{hasRegistered}(simRegistered)) \{
        \textcolor{keywordflow}{return} \textcolor{keyword}{false};
    \} \textcolor{keywordflow}{else} \textcolor{keywordflow}{if} (!simRegistered) \{
        \textcolor{keywordflow}{if} (!\hyperlink{class_u_blox_a2e816e864ebf43743b3f6187e20c2b1f}{startAutoRegistration}(simRegistered)) \{
            \textcolor{keywordflow}{return} \textcolor{keyword}{false};
        \} \textcolor{keywordflow}{else} \textcolor{keywordflow}{if} (!simRegistered) \{
            \textcolor{keywordflow}{return} \textcolor{keyword}{false};
        \}
    \}

    \textcolor{comment}{// Check if GPRS is attached.}
    \textcolor{keywordtype}{bool} gprsAttached = \textcolor{keyword}{false};
    \textcolor{keywordflow}{if} (!\hyperlink{class_u_blox_a4f5a31b4ddda664b255ce3f63e9ffac7}{hasGPRS}(gprsAttached)) \{
        \textcolor{keywordflow}{return} \textcolor{keyword}{false};
    \}

    \textcolor{comment}{// Check if there is an internet connection.}
    \textcolor{keywordtype}{bool} psdConnected = \textcolor{keyword}{false};
    std::string psdUrc;
    \textcolor{keywordflow}{if} (!\hyperlink{class_u_blox_ae49b51a602a327b5eff5b04d2ccaec20}{hasPSD}(psdConnected)) \{
        \textcolor{keywordflow}{return} \textcolor{keyword}{false};
    \} \textcolor{keywordflow}{else} \textcolor{keywordflow}{if} (!psdConnected || !gprsAttached) \{
        \textcolor{keywordflow}{if} (!\hyperlink{class_u_blox_ac250bd4aea14e09b3a2595c2b8eda18a}{connectPSD}(psdConnected, psdUrc) || !psdConnected) \{
            \textcolor{keywordflow}{return} \textcolor{keyword}{false};
        \}
    \}

    \textcolor{comment}{// Configure the sending of messages.}
    \textcolor{keywordflow}{if} (!\hyperlink{class_u_blox_a12c1042d3bcb503b025927fd53d54243}{setSendMessageMode}(\hyperlink{_u_blox_8h_a4fdc1adf2ea333d6490119160a35401a}{SEND\_TEXT\_MODE\_TEXT})) \{
        \textcolor{keywordflow}{return} \textcolor{keyword}{false};
    \}

    \textcolor{comment}{// Configure the scan mode for obtaining the location.}
    \textcolor{keywordflow}{if} (!\hyperlink{class_u_blox_aabed44fd41e16c9d1a8daba80f3bef06}{setLocationScanMode}(\hyperlink{_u_blox_8h_a5c819e4d40995d2854dc0e2cddd7ddef}{LOC\_SCAN\_MODE\_DEEP})) \{
        \textcolor{keywordflow}{return} \textcolor{keyword}{false};
    \}

    \textcolor{keywordflow}{return} \textcolor{keyword}{true};
\}
\end{DoxyCode}
 \paragraph*{Reading and Writing Commands}

Writing AT commands to the device is relatively straight forward -\/ the underlying \hyperlink{_u_art_8h}{U\+Art.\+h} / .cpp \textquotesingle{}.write\+Next(...)\textquotesingle{} function is invoked followed by a \textquotesingle{}read\+Next(...)\textquotesingle{} in order to obtain the echo.


\begin{DoxyCode}
ssize\_t \hyperlink{class_u_blox_af604d1897a66192bf1c2a11997f2634d}{UBlox::writeCommand}(\textcolor{keyword}{const} \textcolor{keywordtype}{char} *command)
\{
    \textcolor{comment}{// Write the command to the device.}
    ssize\_t rc = \hyperlink{class_u_blox_a034c0463d1c199d094d657c8ebb151e8}{uArt}.\hyperlink{class_u_art_aad1ddb133fe430a92527584eec2e674f}{writeNext}(command);
    \textcolor{keywordflow}{if} (rc == -1) \{
        \textcolor{keywordflow}{return} -1;
    \}

    \textcolor{comment}{// Read the raw echo response and check lengths to determine if echoed.}
    rc = \hyperlink{class_u_blox_ab4a7ab4b8922d91e23f273ae160c1bed}{readRawResponse}(\hyperlink{_u_blox_8cpp_a55092c0742d15bb08a5ea7db5a25440e}{RX\_TIMEOUT\_ECHO});
    \textcolor{keywordflow}{if} (rc == -1) \{
        \textcolor{keywordflow}{return} -1;
    \}
    \textcolor{keywordflow}{return} strlen(command) == rc;
\}
\end{DoxyCode}


Given that there is a single device, responses are read into a common buffer in order to preserve memory. This is achieved by one of the two following functions\+:


\begin{DoxyCode}
ssize\_t \hyperlink{class_u_blox_ab4a7ab4b8922d91e23f273ae160c1bed}{UBlox::readRawResponse}(\textcolor{keywordtype}{int} timeoutMs)
\{
    \textcolor{comment}{// Clear the buffer and read the response from the device.}
    \hyperlink{class_u_blox_afc846fbcb1cbd49057b5ce39cd0e0dd6}{clearResponseBuff}();
    \textcolor{keywordflow}{return} \hyperlink{class_u_blox_a034c0463d1c199d094d657c8ebb151e8}{uArt}.\hyperlink{class_u_art_aa4818ca67447e251680b4b8d28c8bba5}{readNext}(\hyperlink{class_u_blox_a6ca4b90f3dc4e856181dce1ebda6f82c}{buffer}, \hyperlink{_u_blox_8h_aad458adf8f40cbcc1074061f226a112e}{AT\_CMD\_BUFF\_LEN}, timeoutMs);
\}
\end{DoxyCode}


or


\begin{DoxyCode}
\textcolor{keyword}{const} \textcolor{keywordtype}{char}* \hyperlink{class_u_blox_a4eaca5b1b1c4b5b6f6164b220dd43e0b}{UBlox::readStatusResponse}(\textcolor{keywordtype}{bool} crlf)
\{
    \textcolor{comment}{// Occasionally there are preceding \(\backslash\)r\(\backslash\)n; read and discard.}
    ssize\_t rc = -1;
    \textcolor{keywordflow}{if} (crlf) \{
        rc = \hyperlink{class_u_blox_ab4a7ab4b8922d91e23f273ae160c1bed}{readRawResponse}(\hyperlink{_u_blox_8cpp_ab6426fc74901f4fbec94862ebb672b81}{RX\_TIMEOUT});
        \textcolor{keywordflow}{if} (rc != 2) \{
            \textcolor{keywordflow}{return} \textcolor{keyword}{nullptr};
        \}
    \}

    \textcolor{comment}{// Generically read the device response and attempt to resolve the status.}
    rc = \hyperlink{class_u_blox_ab4a7ab4b8922d91e23f273ae160c1bed}{readRawResponse}(\hyperlink{_u_blox_8cpp_afed44347eb1fde151258e73004078c98}{RX\_TIMEOUT\_STATUS});
    \textcolor{keywordflow}{if} (rc == -1) \{
        \textcolor{keywordflow}{return} \textcolor{keyword}{nullptr};
    \}
    \textcolor{keywordflow}{return} \hyperlink{class_u_blox_aab6ad68e4c7522278f19ceab1dc2a58d}{checkResponseBuffStatus}();
\}
\end{DoxyCode}


The former merely clears (sets all elements to null terminators \textquotesingle{}\textbackslash{}0\textquotesingle{}) the buffer and reads the raw response, leaving the contents to be interpreted by the invoking function. The latter, however, implicitly invokes the former function with the addition of attempting to resolve the response (\href{https://github.com/itsBelinda/ENG5220-2020-Team13/wiki/uBlox-and-CellLocate}{\tt OK, E\+R\+R\+OR, or A\+B\+O\+R\+T\+ED}), consequently retuning a pointer to the state or nullptr in the event the status could not be resolved.

\subsubsection*{\hyperlink{_u_art_8h}{U\+Art.\+h} / .cpp}

Finally, at the bottom of the stack, the \hyperlink{_u_art_8h}{U\+Art.\+h} / .cpp files are responsible for managing the serial connection with the device. Thus, it implements the means by which commands are written to and read from the device.

First and foremost, the serial interface is initialised i.\+e. the device path (/dev/tty\+S0) is opened and shortly configured using termios.
\begin{DoxyItemize}
\item Due to varying timeout durations associated with each command, V\+M\+IN and V\+T\+I\+ME equal 0, thereby creating a non-\/blocking serial connection.
\end{DoxyItemize}

\paragraph*{Reading}

Instead, commands are read utilising a polling based on the works posited on \href{https://www.i-programmer.info/programming/cc/10027-serial-c-and-the-raspberry-pi.html}{\tt Serial C and The Raspberry PI}. While the interface {\itshape does} implement \textquotesingle{}read\+Expected(...)\textquotesingle{} function for reading until blocks of known sizes, it is not utilised due to surrounding uncertainties of the environment (a premature A\+B\+O\+R\+T\+ED may be returned instead).

Thus, exploiting the format according to which responses are returned (~\newline
 at the end of each response), the \textquotesingle{}read\+Next(...)\textquotesingle{}, the function permits variable sized blocks to be read.


\begin{DoxyCode}
ssize\_t \hyperlink{class_u_art_aa4818ca67447e251680b4b8d28c8bba5}{UArt::readNext}(\textcolor{keywordtype}{char} * \textcolor{keyword}{const} resultBuffer, \textcolor{keyword}{const} \textcolor{keywordtype}{size\_t} resultBufferLen,
                       \textcolor{keyword}{const} \textcolor{keywordtype}{int} timeoutMs)
\{

    \textcolor{comment}{// Check that the device is present.}
    \textcolor{keywordflow}{if} (\hyperlink{class_u_art_a61fb55cc7c92c85f2219dffcfb58bc12}{device} == -1) \{
        \textcolor{keywordflow}{return} -1;
    \}

    \textcolor{comment}{// The number of bytes that have been peeked and read.}
    \textcolor{keywordtype}{size\_t} bytesPeeked = 0;
    ssize\_t bytesRead = 0;

    \textcolor{comment}{// The next and last read indexes.}
    \textcolor{keywordtype}{size\_t} nextReadIndex = 0;
    \textcolor{keywordtype}{size\_t} lastReadIndex = 0;

    \textcolor{comment}{// The last read character.}
    \textcolor{keywordtype}{char} lastReadChar;

    \textcolor{comment}{// Timeout pause.}
    \textcolor{keyword}{struct }timespec timeoutPause = \{0\};
    timeoutPause.tv\_sec = timeoutMs / 1000;
    timeoutPause.tv\_nsec = (timeoutMs % 1000) * 1000000L;

    \textcolor{comment}{// Keep reading the buffer until crlf.}
    \textcolor{keywordflow}{for} (;;) \{

        \textcolor{comment}{// If the buffer has been exceeded, return -1.}
        lastReadIndex = nextReadIndex;
        \textcolor{keywordflow}{if} (lastReadIndex >= resultBufferLen) \{
            \textcolor{keywordflow}{return} -1;
        \}

        \textcolor{comment}{// If there are no characters within the buffer, sleep.}
        ioctl(\hyperlink{class_u_art_a61fb55cc7c92c85f2219dffcfb58bc12}{device}, FIONREAD, &bytesPeeked);
        \textcolor{keywordflow}{if} (bytesPeeked <= 0 && nanosleep(&timeoutPause, \textcolor{keyword}{nullptr})) \{
            \textcolor{keywordflow}{return} -1;
        \}

        \textcolor{comment}{// Read in a single character from the serial buffer.}
        bytesRead = read(\hyperlink{class_u_art_a61fb55cc7c92c85f2219dffcfb58bc12}{device}, &lastReadChar, 1);
        \textcolor{keywordflow}{if} (bytesRead == -1) \{
            \textcolor{keywordflow}{return} -1;
        \} \textcolor{keywordflow}{else} \textcolor{keywordflow}{if} (bytesRead == 1) \{

            \textcolor{comment}{// Write the character to the buffer, check if '\(\backslash\)n'.}
            resultBuffer[nextReadIndex++] = lastReadChar;
            \textcolor{keywordflow}{if} (lastReadChar == \textcolor{charliteral}{'\(\backslash\)n'}) \{
                \textcolor{keywordflow}{break};
            \}

            \textcolor{comment}{// Skip to the next character.}
            \textcolor{keywordflow}{continue};
        \}

        \textcolor{comment}{// If no new bytes have been read and we have timed out, return.}
        \textcolor{keywordflow}{if} (nextReadIndex == lastReadIndex) \{
            \textcolor{keywordflow}{return} -1;
        \}
    \}

    \textcolor{comment}{// Return the number of characters that have been read.}
    \textcolor{keywordflow}{return} nextReadIndex;
\}
\end{DoxyCode}
 Functionally, the \textquotesingle{}read\+Next(...)\textquotesingle{} utilises C\textquotesingle{}s nanosleep for inter-\/byte timeouts to sleep the thread until it is awoken due to the device responding with bytes; this state is entered in the event all characters have been read from the underlying Linux / device buffer (any bytes within the buffer will be immediately processed). Thus, the algorithm is functionally analogous to polling with the said nanosleep function providing the means by which inter-\/character timeouts are achieved.

\paragraph*{Writing}

$\ast$$\ast$\+Assuming that the command conforms to the defined format i.\+e. ending in , the command can be written to the device by invoking the following function.$\ast$$\ast$ 
\begin{DoxyCode}
ssize\_t \hyperlink{class_u_art_aad1ddb133fe430a92527584eec2e674f}{UArt::writeNext}(\textcolor{keyword}{const} \textcolor{keywordtype}{char} *command)
\{
    \textcolor{keywordflow}{if} (\hyperlink{class_u_art_a61fb55cc7c92c85f2219dffcfb58bc12}{device} != -1 && tcflush(\hyperlink{class_u_art_a61fb55cc7c92c85f2219dffcfb58bc12}{device}, TCIFLUSH) == 0) \{
        \textcolor{keywordflow}{return} write(\hyperlink{class_u_art_a61fb55cc7c92c85f2219dffcfb58bc12}{device}, command, strlen(command) + 1);
    \}
    \textcolor{keywordflow}{return} -1;
\}
\end{DoxyCode}
 Functionally, the function ensures that the device is present, flushes the data already written to the device and writes the command to the underlying device buffer. \subsection*{Running Program}

 