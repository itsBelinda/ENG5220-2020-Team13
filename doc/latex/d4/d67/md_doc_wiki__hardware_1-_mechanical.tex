Due to unforeseen circumstances, the hardware part of the project could not be properly finished.

The current setup looks like this (in this picture, a keyboard and monitor was connected to the PI, this is not necessary)\+: 

 

\section*{Next Steps}

To test the setup when moving, a casing was designed using Autodesk Inventor. The files can be found in the \href{https://github.com/itsBelinda/ENG5220-2020-Team13/blob/master/hardware/mechanical/case}{\tt {\ttfamily mechanical}} folder. They include the Autodesk Inventor files as well as \+\_\+.\+stp\+\_\+ files that are needed to 3D print the casing.

The casing only has some slits for air, a hole for the U\+SB power supply and one for the antenna. In contrast to other Raspberry PI cases that have all the ports accessible, this case is a little bigger and has the ports closed off.

The board can then be supplied by a mobile power bank. This power bank should have the \char`\"{}quick charge\char`\"{} feature (can supply up to 2 A). Otherwise the standard U\+SB current output of max. 0.\+5 A might not be enough when sending data.

\subsection*{The Casing}

This casing has never been printed, so rendered pictures are shown. 

 

\subsection*{Top and Bottom}

  