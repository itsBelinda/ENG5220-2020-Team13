Two main means of documentation have been created for this project, for two distinct applications\+: Github Wiki and Doxygen.

\subsection*{Github Wiki}

This Github wiki describes the project as a whole -\/ its motivations, uses, components, and how it was carried out -\/, and is split into 5 chapters\+: ~\newline

\begin{DoxyItemize}
\item The $\ast$$\ast$\href{https://github.com/itsBelinda/ENG5220-2020-Team13/wiki/Build-Guide}{\tt Build guide}$\ast$$\ast$ describes how our project can be recreated, or installed for expanding on/contributing to it.
\item The $\ast$$\ast$\href{https://github.com/itsBelinda/ENG5220-2020-Team13/wiki/User-Guide}{\tt User guide}$\ast$$\ast$ explains how to run the project as a developer and how to use it as a parent.
\item The $\ast$$\ast$\href{https://github.com/itsBelinda/ENG5220-2020-Team13/wiki/Hardware}{\tt Hardware chapter}$\ast$$\ast$ details the design, construction and testing processes of the hardware required for the project\+: the electronic components that make up the P\+CB connected to the Pi, the casing that did not end up being produced due to the C\+O\+V\+I\+D19 lockdown of the university facilities, and describes the communications module and its functionality in depth.
\item The $\ast$$\ast$\href{https://github.com/itsBelinda/ENG5220-2020-Team13/wiki/Software}{\tt Software chapter}$\ast$$\ast$ discusses the underlying software components of the project. It describes the design and build of the code running on the Pi communicating with the u-\/blox module and the server, the web interface available for parent, as well as the testing and integration settings for the program.
\item The $\ast$$\ast$\href{https://github.com/itsBelinda/ENG5220-2020-Team13/wiki/Project-Management}{\tt Project Management}$\ast$$\ast$ chapter reflects on how the project was carried out, and any adjacent, administrative tasks are discussed here. This includes how the marketing of the project was carried out, how the team worked, how others can contribute to the project if interested, how the project was documented, and provides an evaluation of the final product, mentioning future plans for the Bee\+Safe project and adding handy tips for those planning to carry out a similar embedded project.
\end{DoxyItemize}

\subsection*{Doxygen}

Doxygen is an automated code documentation tool. It produces an interactive, navigable H\+T\+ML documentation for the project software, listing what classes and functions do, their parameters and types, and highlights the connections between members that build up the program. It can be found \href{https://itsbelinda.github.io/ENG5220-2020-Team13/doc/html/index.html}{\tt here}. 